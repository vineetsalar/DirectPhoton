%\documentstyle[aps,preprint,psfig]{revtex}
\documentstyle[prc,psfig,aps,twocolumn]{revtex}

\begin{document}
\twocolumn[
\hsize\textwidth\columnwidth\hsize\csname @twocolumnfalse\endcsname

\draft
\begin{center}
PHYSICAL REVIEW C (submitted)
\end{center}

\title{A first order quark-hadron phase transition
            and SPS photon spectrum}
\author{Prashant Shukla}
\address{Nuclear Physics Division,\\
 Bhabha Atomic Research Center, Mumbai 400085, India}
\maketitle

\begin{abstract}
 Various scenarios of a first order quark hadron phase transition and 
their manifestations in the photon 
spectrum are investigated, in a hydrodynamical model. These scenarios 
arising from supercooling have been used to calculate the photon spectrum.
 The calculated photon 
spectra are compared with the photon data measured in Pb+Pb collision 
experiments at $\sqrt s = 17.6A$ GeV at SPS, CERN. 
 Imprints of these scenarios on the other measured data 
is also discussed.

\vspace{.1in}  
\noindent
{\bf PACS numbers}: 12.38.Mh, 25.75.-q, 05.70.Fh
\end{abstract}
]

\narrowtext


\vspace{.2in}  

  The ultra relativistic heavy ion collision experiments are performed 
with the aim to produce matter with very high energy density required to 
form quark gluon plasma (QGP). With this aim, Pb+Pb collision experiments 
at $\sqrt s = 17.6A$ GeV have been carried out at CERN SPS giving a 
variety of data on electromagnetic and hadronic observables
[For a review see~\cite{ESKOLA}]. 
  The most interesting of these are the electromagnetic 
probes~\cite{WA98,CERES}, 
which are sensitive to various stages of the evolution of the system and 
may contain the information on QGP formation. 
  A considerable amount of theoretical work has been done
to extract the information on QGP from these 
data [For review see~\cite{PEITTHOMA,GALER}].

  To explain the data with hydrodynamical models, one assumes that, QGP 
is formed with given initial conditions which then expands
hydrodynamically till it reaches the critical temperature $T_C$ for 
a first order quark-hadron phase transition.
  In the idealized Maxwell construction, the temperature of the system 
is held fixed at $T=T_C$ until the hadronization is completed. 
Thus, in turn one assume that the latent heat obtained by the phase 
transition compensates the temperature drop due to cooling.
Such a picture has been widely used in literature while describing
the electromagnetic signals (See e.g.~\cite{SRIVASTAVA}).
  This is a good approximation when the expansion time scales are far 
greater than the phase transition time scales as in the case of early 
universe. 
  For the case of relativistic heavy ion collisions, there have
been extensive studies of the space time evolution of 
QGP undergoing a first order phase transition by 
nucleation~\cite{CSER,SHUK,INHOMO,ZABPRC}.
  As a result of rapid expansion of the system, a substantial amount of 
supercooling is expected and the dynamics of quark hadron phase 
transition becomes quite different in heavy ion collisions as compared to 
that in early universe~\cite{INHOMO,AKM}. 
 When the hydrodynamical equations are coupled with nucleation rate 
equation~\cite{SPINO}, the QGP is shown to supercool to a temperature 
$T_S$ lower than $T_C$. The hadronization proceeds by thermal fluctuation
from this metastable QGP to stable hadron phase. The system 
reheats due to the release of latent heat and approaches $T_C$ as 
the hadronization proceeds. 
This also corresponds to an equilibrium situation where the latent heat 
generated goes into reheating the system. 

  In heavy ion collisions, the cooling time scales are comparable
to the phase transition time scales, that gives rise to many
interesting possibilities around $T_C$.
  It is possible for example, that QGP supercools to a temperature, 
at which the barrier between the metastable QGP and the stable hadron phase 
minima completely vanishes leading to a point of inflection at 
$T=T_{\rm spino}$. This is known as 
spinodal instability~\cite{SPINO}. In this case, the rapidly quenched 
system leaves the region of metastability and enters the highly unstable 
spinodal region before a substantial amount of nucleation begins.
There is no metastable state from which one can not define the phase 
transition by thermal fluctuation.
   Alternatively, spinodal decomposition has been suggested as a
possible mechanism of phase conversion for a rapidly expanding system of 
quarks and gluons~\cite{SPINO,DUMHEP,DUMPRL}. This hadronization process
is much faster than the nucleation and is  
also referred as explosive decomposition~\cite{DUMEXP}.
 In this scenario, the QGP breaks up into small droplets of plasma in
a non equilibrium way, which will form hadrons. It is unlikely for the 
system to undergo re-equilibration and thus it will freeze out at this point. 
  Such a picture has been presented many times in the literature
and it is referred as `a fast shock like hadronization' in 
Refs.~\cite{CSORGO} and as `a sudden hadronization scenario' in 
Ref.~\cite{RAFEL}.
  The signature of such a scenario will be very short life 
time~\cite{CSERNAI} of the system measured through HBT data
and should combine with no mass shift in $\rho$ and $\phi$ etc.
observed in dilepton spectrum and large departures of hadron spectrum
from equilibrium distribution.

  In case of nucleation, another possibility arises at the late stage of 
hadronization when most of the QGP volume has been converted to hadrons.
%  In full nucleation model one assumes that the already nucleated hadron 
%bubbles join together and the QGP bubbles are trapped inside. 
%Then one can continue with the nucleation model, assuming that these QGP 
%droplets will shrink and the shrinking velocity is equal to the 
%hadron bubble growth velocity. This is a very simplistic picture.
  The system can be considered in the form of hadron clusters and the small 
QGP bubbles. At such a point, fluctuation theory is not applicable for 
hadronization of remaining small fraction of QGP. Further hadronization by 
equilibrium processes stops here and the system may simply break up into 
fragments: hadronic clusters and small droplets of plasma which then form 
the final state hadrons. Such a possibility is presented
in Ref.~\cite{ZABPRC}. 

   As all these scenarios exist in the literature, it is worthwhile 
to see their manifestation in the measured photon spectrum~\cite{WA98}
which is the best probe of space time evolution of 
quark-gluon plasma (QGP) possibly produced in relativistic heavy-ion
collisions. While a number of models have been proposed in context 
with photon data, the supercooling, which is an essential feature of 
first order phase transition has never been included to calculate photon 
spectrum. Here we concentrate mainly on this aspect and 
apply the most common set of parameters, equation of state 
and the expansion scenario for our calculations.
 We consider that a baryon free QGP is formed at some initial temperature 
$T_i=211$ MeV and initial time $\tau_i=0.8$ fm and starts cooling by Bjorken 
hydrodynamics. Transverse expansion is not taken into account considering 
short life time of the system. To get the equations of state, it is assumed 
that QGP is a massless 2 flavour parton gas and hadronic phase is a massless 
three pion gas.
 When the QGP cools to the critical temperature $T_C=160$ MeV at 
$\tau_C=1.83$ fm, we invoke following four scenario for the calculations:

\begin{enumerate}

\item Idealized Maxwell construction:
    The temperature of the system is held fixed at $T=T_C$ 
    during hadronization. The hadronization completion time 
    $\tau_h\sim$ 23 fm can be simply calculated by entropy conservation.
    After this, the hadron gas cools to freeze out temperature 
    $T_f=145$ MeV at $\tau_f\sim 30$ fm. Such a calculation is shown by 
    dashed line in Fig.~1.

\item  Nucleation and supercooling scenario:
   At $T=T_C$, hydrodynamical equations are coupled with nucleation 
  rate equations~\cite{SPINO}. The QGP supercools to a temperature 
  $T_S=0.9T_C$. The hadronization proceeds by 
 thermal fluctuation from this metastable QGP phase to stable hadron phase. 
 The system reheats due to the release of latent heat and approaches
 $T=T_C$ as the hadronization proceeds. When hadronization is completed,  
 the hadron gas cools and freezes out at $\tau_f=34$ fm.
 Such a calculation is shown by solid line in Fig.~1.

\item Spinodal decomposition followed by freeze out:
  QGP supercools to the temperature $T=T_S$, which is close to spinodal 
  temperature~\cite{SPINO} which can be calculated as 
\begin{equation} 
T_{\rm spino}= \left[{B \over B+ 27 V_b} \right]^{1/4} \, T_C, 
  \hspace{.5in} V_b = {3\sigma \over 16 \xi}.
\end{equation}
For Bag constant $B^{1/4}$= 222 MeV,
surface tension $\sigma$= 25 MeV/fm$^2$ and correlation length $\xi$ = 0.7 fm,
one can calculate 
$T_{\rm spino}=0.894 T_C$, which is very close to the minimum temperature
in supercooling. 
 The QGP breaks up into small droplets of plasma, which will form 
hadrons coinciding with freeze out at temperature $T_S$ and
time $\tau_s=$ 2.93 fm.


\item Fragmentation at late stage of hadronization followed by freeze out:

  In the nucleation scenario when most of the QGP volume has been 
converted to hadrons at time $\tau_p$, the system breaks up into fragments: 
hadronic clusters and small droplets of plasma. 
 The time $\tau_p$ is adjusted to reproduce the SPS data. This 
corresponds to 94 \%  hadronization at time $\tau_p=$ 15.5 fm.
This is also assumed to coincide with freeze out.
\end{enumerate}
 It is to be mentioned that, the same supercooling curve is used for 
scenario 2, 3 and 4. They differ only by implementation of freeze
out at different times, due to different physical conditions.

\begin{figure}[t]
\centerline{\psfig{figure=fig1.ps,width=8.5cm,height=8cm}}
\caption{The idealized Maxwell construction and supercooling curves.}
\end{figure}


  The total thermal photon emission rate is given by the sum of 
corresponding rates $R_{\rm QGP}$ in the QGP and $R_{\rm hadron}$ 
in the hadron regions integrated over space time volume 
element $d^4x = \tau d\tau d\eta_f \pi R^2$ as
\begin{eqnarray}\label{}
 E \frac{dN}{d^3 p}  &=& \int_{\tau_i}^{\tau_f} \int_{-l}^{l} 
     [(1-h(\tau)) R_{\rm QGP}(T(\tau),\eta_f) \nonumber \\
  &+& h(\tau)  R_{\rm hadron}(T(\tau),\eta_f) ]d\eta_f \tau d\tau \pi R^2.
\end{eqnarray}
Here $\eta_f$ is the fluid rapidity integrated from 
$-l$ to $+l$ ($l=2.8$) and $R$ is the size of colliding 
nuclei given by $R=1.2\, A^{1/3}$, A=208. 
 The temperature, $T(\tau)$ and the hadronic fraction, $h(\tau)$ are
obtained as a function of time using the four scenarios mentioned above. 

  For $R_{\rm QGP}$ and $R_{\rm hadron}$, we use the parameterized forms
given in Ref.~\cite{THOMA}. The QGP rate corresponds to 
the photon production rate from an equilibrated 2 flavour
QGP, calculated in perturbative thermal QCD.
  The contribution of prompt photons, produced in the initial 
scattering has been obtained by the PYTHIA
calculations of Ref.~\cite{GALLMEISTER} and are parameterized as
\begin{eqnarray}\label{promp}
E \frac{dN}{d^3 p} = \exp(a + bP_T + cP_T^2 + dP_T^3), \,\, {\rm with} \nonumber\\
  a=-5.9883,\,\, b=2.0934,\,\, c=-1.6015,\,\, d=0.1432.
\end{eqnarray}



   Figure~2 demonstrates how the thermal photon contributions due to 
different evolutions compare with SPS data (filled circles). The prompt 
photons (thick solid line) have been added in all calculations.
 The idealized Maxwell construction corresponds to the short dashed line.
It exceeds the experimental limits in the low $P_T$ region.
The full nucleation and supercooling (scenario 2) corresponds to 
the middle dashed line which shows more photons in the low 
$P_T$ region. This is due to the fact that the system has spent more
time in region with temperature lower than $T_C$. 
  The scenario (3) (long dashed line) looks the most improbable at SPS 
energies. The analysis of CERES dilepton data shows signatures of 
in-medium effects and mass shifts of $\rho$ and $\phi$ mesons~\cite{RAPP} 
supporting above conclusion. 

\begin{figure}[t]
\centerline{\psfig{figure=fig2.ps,width=8.5cm,height=8cm}}
\caption{The SPS photon data and the calculations of hydrodynamical
model using various scenarios of first order phase transition.}
\end{figure}

  From this figure (Fig.~2), scenario (4) (solid line) seems the 
most appropriate evolution scenario at SPS. 
The freeze out temperature is $\sim 160$ MeV. 
This gives a life time of $\sim 15$ fm which is about half of the
life time given by the idealized Maxwell construction. It should be 
judged against the HBT data, which show short duration of particle 
emission as well as short life time of the system before freeze 
out~\cite{HBT}. The measured hadron yields
must increase at higher momentum over the value obtained from the 
equilibrium distributions, due to the hadrons coming from the 
QGP droplets after freeze out. Notably, in the analysis of SPS hadron 
spectrum in Ref.~\cite{XU}, it was observed that there are more hyperons 
than expected, which could result from the picture presented above.

  We would like to mention that it is also possible to reproduce the SPS
photon data by different models as reviewed in~\cite{PEITTHOMA,GALER},
some of which we discuss in the following:
 Taking the scenario 1 as baseline, the photon data shows a much flatter
distribution or in other words a larger slope. One can achieve a flatter
distribution by the scenario 1 by different means.
 In the most popular analysis by 
Srivastava et al.~\cite{SRIVASTAVA}, effective degrees of freedom in 
the hadron gas is increased. This reduces the life time of the 
mixed phase thereby reducing the photon contribution in low $P_T$ region.
In addition, tougher initial conditions $\tau_i = 0.2$ fm and 
$T_i = 335$ MeV, are used to enhance the
photon contribution in high $P_T$ region. In the work of 
Alam et al.~\cite{ALAM}, it is achieved by considering an initial radial 
velocity and in medium modification of hadron masses. In the work of 
Gallmeister~\cite{GALLMEISTER}, prompt photon contribution and
transverse flow effect have been taken into account. 
Huovinen and Russkanen~\cite{HUOV} have introduced a strong flow at later 
stages to explain the data. All these 
work~\cite{SRIVASTAVA,ALAM,GALLMEISTER,HUOV} point to a short lifetime
of mixed phase at SPS. 
 
  In conclusions, we have presented the results of investigations on 
different scenarios of a first order phase transition and analyzed
SPS photon spectrum. The Supercooling which is an essential 
feature of first order phase transition has been included in the work.
We find that the effect of supercooling is to marginally increase the 
photons in the low $P_T$ region. A sudden hadronization scenario is ruled 
out at SPS. The so called fragmentation followed by freeze out seems 
to be the most appropriate scenario at SPS. This scenario supports a 
shorter life time of mixed phase at SPS also indicated by the previous
analyses.
  The transverse expansion and finite baryon density have been neglected
in the present work. The effect of finite baryon density is to reduce
photons in high $P_T$ region and increase in low $P_T$ 
region [see discussions in~\cite{PEITTHOMA}] which is opposite to the
effect of transverse expansion. 
Thus, these two effects tend to cancel each other.
However, these will be included in a future communication. 
 A more detailed hydrodynamical calculation including fragmentation is to 
be carried out to analyses the data on photon, dilepton and HBT radii 
measured at SPS and also at RHIC.

\acknowledgements
 The author acknowledges the stimulating discussions with A.K.
Mohanty, D.K. Srivastava and Z. Ahmed.


\begin{thebibliography}{00}

\bibitem{ESKOLA} K. J. Eskola, {\it High Energy Nuclear Collisions}, 
  Plenary talk given at Int. Europhysics conference on High Energy Physics 
  (EPS-HEP99), Tampere, Finland, July 15-21, 1999, 
  Preprint: hep-ph/9911350 (1999).
  
\bibitem{WA98} WA98 Collab., M.M.\ Aggarwal et al., 
            Phys. Rev. Lett. 85, 3598 (2000); nucl-ex/0006007 (2000).

\bibitem{CERES} B. Lenkeit for CERES Coll., 
        Nucl. Phys. {\bf A654}, 627c (1999); nucl-ex/9910015.  
  
\bibitem{PEITTHOMA} T. Peitzmann and M.H. Thoma, Phys. Rep. 364, 175 (2002);
                    hep-ph/0111114.

\bibitem{GALER} C. Gale and K.L. Haglin, in {Quark Gluon Plasma 3},
    ed. R.C. Hwa and X.-N. Wang, World Scientific, Singapore, 2003;
    hep-ph/0306098. 
          
\bibitem{SRIVASTAVA} D.K.\ Srivastava and B.\ Sinha, 
            Phys. Rev. C{\bf 64}, 034902 (2001). 

\bibitem{CSER} L.P. Csernai and J.I. Kapusta, Phys. Rev. Lett. {\bf 69},
               737 (1992).

\bibitem{SHUK} P. Shukla, S.K. Gupta, and A.K. Mohanty,
         Phys. Rev. C{\bf 59}, 914 (1999); {\it ibid} {\bf 62}, 39901 (2000).

\bibitem{INHOMO} P. Shukla, A.K. Mohanty, S.K. Gupta, and M. Gleiser,
              Phys. Rev. C{\bf 62}, 054904 (2000).

\bibitem{ZABPRC} E.E. Zabrodin, L.V. Bravina, H. Stocker, and W. Griener,
         Phys. Rev. C{\bf 59}, 894 (1999).

\bibitem{AKM} A.K. Mohanty, P. Shukla and M. Gleiser,
              Phys. Rev. C{\bf 65}, 034908 (2002).
              
\bibitem{SPINO} P. Shukla and A. K. Mohanty, Phys. Rev. C{\bf 64},
        054910 (2001).              

\bibitem{DUMHEP} O. Scavenius, A. Dumitru, E.S. Fraga, J.T. Lenaghan,
         A.D. Jackson, Phys. Rev. D{\bf 63}, 116003 (2001).

\bibitem{DUMPRL} O. Scavenius, A. Dumitru,
         Phys. Rev. Lett. {\bf 83}, 4697 (1999).
         
\bibitem{DUMEXP} O. Scavenius, A. Dumitru, A.D. Jackson, 
          hep-ph/0103219.
          
\bibitem{CSORGO} T. Csorgo, L.P. Csernai, Phys. Lett {\bf B333}, 494 (1994);
       L.P. Csernai, I.N. Mishustin, Phys. Rev. Lett. {\bf 74}, 5005 (1995).
       
\bibitem{RAFEL} J. Rafelski and J. Letessier, Phys. Rev. Lett. {\bf 85},
                  4695 (2000).

\bibitem{CSERNAI} L.P. Csernai, M.I. Gorenstein, L.L. jenkovszky, 
                        I. Lovas and V.K. Magas, hep-ph/0210297.

\bibitem{THOMA} F.D. Steffen and M.H. Thoma, Phys. Lett. B510, 
                  1998 (2001).
                  
\bibitem{GALLMEISTER} K.\ Gallmeister, B.\ K\"ampfer, and O.P.\ Pavlenko, 
            Phys. Rev. C{\bf 62}, 057901 (2000); hep-ph/0006134 (2000).

\bibitem{RAPP} R. Rapp and J. Wambach, Euro. Phys. J., A6415 (1999);
           G.Q. Li, C.M. Ko and G.E. Brown, Phys. Rev. Lett. 75,
           4007 (1995).

\bibitem{HBT} M. Gyulassy, nucl-th/0106072; PHENIX Collaboration, 
           K. Adcox et. al., Phys. Rev. Lett. 88. 192302 (2002).

\bibitem{XU} N. Xu and M. Kaneta, Nucl. Phys. {\bf A698}, 306c (2002).

\bibitem{ALAM} J. Alam, S. Sarkar, T. Hatsuda, T.K. Nayak, and B. Sinha,
              Phys. Rev. C{\bf 63}, 021901 (2001).

\bibitem{HUOV} P. Huovinen, P.V. Ruuskanen, and S.S. Rasanen, nucl-th/0111052.
                
\end{thebibliography}






\end{document}




