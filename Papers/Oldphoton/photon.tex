\documentstyle[12pt,psfig]{article}
\textwidth=15cm
\textheight=22cm
%\topmargin=-2.5cm
%\oddsidemargin=-.3cm
\pagestyle{empty}
\begin{document}

\begin{center}
{\large \bf Effect of dynamics of quark-hadron phase transition
            on SPS and RHIC results}\\
{\bf P. Shukla}\\
{\it Nuclear Physics Division,\\
 Bhabha Atomic Research Center, Mumbai 400085, India}
\end{center}


\section{Introduction}

\section{The photon distribution}
  The photon emission rate is given by the sums of corresponding 
rates in the QGP and hadron regions as 
\begin{eqnarray}\label{}
E {dN \over d^4x d^3p} = 
 (1-h) \left(E {dN \over d^4x d^3p}\right)_{\rm QGP} +
  h  \left(E {dN \over d^4x d^3p}\right)_{\rm hadron}.
\end{eqnarray}
where $\eta$ is the rapidity of the medium with temperature $T$.

Integrating on the space time
\begin{eqnarray}\label{}
E {dN \over d^3p} &=& \int \left[(1-h) R_{\rm QGP} (T(\tau),\eta)+ 
     h  R_{\rm hadron}(T(\tau),\eta) \right] d^4 x, \nonumber \\
   &=& \int_{-l}^{l} \int_{\tau_C}^{\tau_f}
      \left[(1-h) R_{\rm QGP}(T(\tau),\eta) 
   + h  R_{\rm hadron}(T(\tau),\eta) \right]\tau d\tau d\eta \pi R^2
\end{eqnarray}
The temperature $T(\tau)$ is obtained from the hydrodynamic equations
\begin{eqnarray}\label{}
T(\tau)  &=& T_C  \hspace{.5in} 0 < h < 1   \nonumber \\
      &=& (1-h) T_0 \left({\tau_0 \over \tau}\right)^{1/3}
        + h T_C \left({\tau_h \over \tau}\right)^{1/3},
           \hspace{.5in} h=0 {\rm or}  h = 1.
\end{eqnarray}

 The photon emission rate in the QGP sector considering 
the processes ($qg\rightarrow q\gamma$),
($q \bar q \rightarrow g \gamma$)
is given by 
\begin{eqnarray}\label{}
E {dN \over d^4x d^3p} = 0.0281 {18\pi^2\over 5} \alpha \alpha_s 
         \ln\left({0.23 E \over \alpha_s T} \right) T^2 
         \exp\left(-\frac{E}{T}\right).
\end{eqnarray}

 The photon emission rate in the hadron sector considering the 
processes 
($\pi\pi\rightarrow \rho\gamma$),
($\pi\rho \rightarrow \pi\gamma$),
($\omega \rightarrow \pi\gamma$),
($\rho \rightarrow \pi\pi\gamma$),
is given by 

\begin{eqnarray}\label{}
E {dN \over d^4x d^3p} = 4.8 T^{2.15}  
      \exp\left(-\frac{1}{(1.35TE)^{0.77}}\right)
                \exp\left(-\frac{E}{T}\right).
\end{eqnarray}
 Here $\alpha=1/137$, and the strong coupling constant is 
\begin{eqnarray}\label{}
\alpha_s= \frac{6\pi}{(33-2N_f) \ln(8T/T_C)}.
\end{eqnarray}

\begin{figure}
\centerline{\psfig{figure=photo.ps,width=12cm,height=10cm}}
\caption{}
\label{photon}
\end{figure}

\begin{figure}
\centerline{\psfig{figure=photop.ps,width=12cm,height=10cm}}
\caption{}
\label{photon}
\end{figure}



\section{The dilepton distribution}
  The dilepton emission rate is given by the sums of corresponding 
rates in the QGP and hadron regions as 
\begin{eqnarray}\label{}
E {dN \over d^4x d^3p dM^2} = 
 (1-h) \left(E {dN \over d^4x d^3p dM^2}\right)_{\rm QGP} +
  h  \left(E {dN \over d^4x d^3p dM^2}\right)_{\rm hadron}.
\end{eqnarray}
where $M$ is the invariant mass of the lepton pair and $\eta$ is the 
rapidity of the medium with temperature $T$.

Integrating on the space time
\begin{eqnarray}\label{}
E {dN \over d^3p dM^2} &=& \int \left[(1-h) R_{\rm QGP} (T(\tau),\eta)+ 
     h  R_{\rm hadron}(T(\tau),\eta) \right] d^4 x, \nonumber \\
   &=& \int_{-l}^{l} \int_{\tau_C}^{\tau_f}
      \left[(1-h) R_{\rm QGP}(T(\tau),\eta) 
   + h  R_{\rm hadron}(T(\tau),\eta) \right]\tau d\tau d\eta \pi R^2
\end{eqnarray}
The temperature $T(\tau)$ is obtained from the hydrodynamic equations
\begin{eqnarray}\label{}
T(\tau)  &=& T_C  \hspace{.5in} 0 < h < 1   \nonumber \\
      &=& (1-h) T_0 \left({\tau_0 \over \tau}\right)^{1/3}
        + h T_C \left({\tau_h \over \tau}\right)^{1/3},
           \hspace{.5in} h=0 {\rm or}  h = 1.
\end{eqnarray}

 The dilepton emission rate in the quark sector considering the 
processes ($q \bar q \rightarrow e^- e^+$ and 
$q \bar q \rightarrow \mu^- \mu^+$)
is given by 

\begin{eqnarray}\label{}
E {dN \over d^4x d^3p dM^2}|_{\rm QGP}
&=& \frac{9}{(2\pi)^5}  \left(\sum e_q^2\right) M^2 \sigma(M^2)
        \exp\left(-\frac{E}{T}\right), \nonumber \\
&=& \frac{\alpha^2}{8\pi^4} \left(\sum e_q^2\right) 
        \exp\left(-\frac{E}{T}\right).
\end{eqnarray}
Here, $\sigma(M^2)=4\pi\alpha^2/9M^2$ and
$\sum e_q^2 =  (1/3)^2 + (2/3)^2 =5/9$.

 The dilepton emission rate in the hadron sector considering the 
processes ($\pi^- \pi^+ \rightarrow \rho \rightarrow e^- e^+$)
is given by 
\begin{eqnarray}\label{}
E {dN \over d^4x d^3p dM^2}|_{\rm hadron}
 = \frac{\alpha^2}{8\pi^4} F_h
        \exp\left(-\frac{E}{T}\right).
\end{eqnarray}
The factor $F_h$ is given by
\begin{eqnarray}\label{}
F_h = \frac{1}{12} \frac{m_\rho^4}{(m_\rho^2-M^2)^2 + m_\rho^2 \Gamma_\rho^2}
\end{eqnarray}
The parameters used are $m_\rho$=775, $\Gamma$=151.


\section{HBT radii and other parameters at SPS and RHIC}
At SPS 158A GeV Pb + Pb collisions at center of mass energy 
17.5A GeV the various parameters are as follows:

$N_{ch}= 700$ \\
Dynamical fluctuations of $<P_T>$ $\leq$ 3 \% \\
$\eta \simeq \pm 1$ \\
$\beta_c$ = 0.55 \\
$\mu_B$= 270 MeV \\
$T_f$ = 143 MeV \\

At RHIC Au + Au collisions at center of mass energy 
130A GeV the various parameters are as follows:

$N_{ch}= 4000$ \\
Dynamical fluctuations of $<P_T>$ $\leq$ 8 \% \\
$\eta \simeq \pm 1.5$ \\
$\beta_c$ = 0.76 \\
$\mu_B$= 50 MeV \\
$T_f$ = 140 MeV \\

HBT radii from PHOBOS experiment are 5-7 fm.
The ratio $R_{out}/R{side} \leq 1$. 
On the basis of hydrodynamics $R_{out}/R{side} \sim 2$



\end{document}

