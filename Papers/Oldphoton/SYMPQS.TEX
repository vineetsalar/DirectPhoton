\documentstyle[12pt,psfig]{article}
\textwidth=14cm
\textheight=20cm
\pagestyle{empty}
\begin{document}

\begin{center}
{\large \bf Dynamics of quark-hadron phase transition and 
  CERN SPS photon spectrum}\\
Prashant Shukla  \\
{\it Nuclear Physics Division, 
Bhabha Atomic Research Centre,\\
Trombay, Mumbai 400 085, India}\\
\end{center}

  The ultra relativistic heavy ion collision experiments are performed 
with the aim to produce matter with very high energy density required to 
form quark gluon plasma (QGP). With this aim, Pb+Pb collision experiments 
at $\sqrt s = 17.6A$ GeV have been carried out at CERN SPS giving a 
variety of data on electromagnetic and hadronic observables.
  The most interesting are the electromagnetic probes \cite{WA98}, 
which are sensitive to various stages of the evolution of the system and 
may contain the information about QGP formation. 

  To explain the data with hydrodynamical models, one assumes that QGP 
is formed with given initial conditions which then expands
hydrodynamically till it reaches the critical temperature $T_C$ for 
quark-hadron phase transition.
  In the idealized Maxwell construction, the temperature of the system 
is held fixed at $T=T_C$ until the hadronization is completed. 
Thus in turn one assume that the latent heat obtained by the phase 
transition is compensating the temperature drop due to cooling.
Such a picture has been widely used in literature when describing
the electromagnetic signals \cite{SRIVASTAVA}.
  This is a good approximation when the expansion time scales are far 
greater than the phase transition time scales as in the case of early 
universe. For the case of relativistic heavy ion collisions, when the
hydrodynamical equations are coupled with nucleation rate equation 
\cite{INHOMO}, the QGP is shown to supercool to a temperature 
$T_S$ lower than $T_C$. The hadronization proceeds by thermal fluctuation
from this metastable QGP to stable hadron phase. The system 
reheats due to the release of latent heat and approaches to $T_C$ as 
the hadronization proceeds. Such a calculation is shown in Fig.~1.
This corresponds to an equilibrium situation where the latent heat 
generated is going into reheating the system. 

If QGP supercools to a temperature where it goes to an unstable state 
then one can not define the phase transition rate by thermal fluctuation. 
This point is known as spinodal instability \cite{SPINO}. The QGP will
break up into small droplets of plasma which forms hadrons. 
The system may not undergo further reequilibration and will freeze
out at this point. The signature of such a scenario is no mass shift in 
$\rho$ and $\phi$.

  Another possible scenario towards the end of nucleation, when most of 
the QGP volume has been converted to hadrons. The already nucleated hadron 
bubbles join together and the QGP droplets will be trapped inside. 
In such a situation nucleation theory is no more applicable and the 
system simply breaks up into fragments: 
clusters of hadrons and small droplets of plasma which then forms the 
final state hadrons \cite{ZABPRC}. 

  We consider that QGP is formed at some initial temperature $T_i$ and
initial time $\tau_i$ and starts cooling by the Bjorken hydrodynamics.
Transverse expansion is not taken into account considering short 
life time of the system. The equation of state of QGP is taken as 
massless 2 flavour parton gas and massless three pion gas for hadronic phase.
At the critical temperature $T_C$ following scenarios are invoked:


(1) --- Idealized Maxwell construction (from $T_i$ to $T_f$.)
(2) --- Supercooling scenario (from $T_i$ to $T_f$.)
(3) --- Spinodal decomposition followed by freeze out
             (from $T_i$ to $T_S$.)
(4) --- Fragmentation at 95 \% of hadronization followed by freeze out
               (from $T_i$ to $T_P$.)

  The total thermal photon emission rate is given by the sum of 
corresponding rates in the QGP and in the hadron regions integrated 
over space time volume obtained from above four scenario. The photon 
production rates are taken from the parameterized forms of Ref.~\cite{THOMA}
The contribution of prompt photons has been obtained by the PYTHIA
calculations \cite{GALL}.

   In Fig.~2, the contribution of thermal photons are calculated and 
compared with SPS data.

\begin{figure}
\begin{center}
\centerline{\hbox{
\psfig{figure=supern.ps,height=7cm,width=7cm} 
\psfig{figure=photodif.ps,height=7cm,width=7cm}}}
\end{center}
\end{figure}


\begin{thebibliography}{00}

\bibitem{WA98} WA98 Collab., M.M.\ Aggarwal et al., 
            Phys. Rev. Lett. 85, 3598 (2000); nucl-ex/0006007 (2000).

\bibitem{SRIVASTAVA} D.K.\ Srivastava and B.\ Sinha, nucl-th/0006018 (2000).

\bibitem{INHOMO} P. Shukla, A.K. Mohanty, S. K. Gupta, and M. Gleiser,
              Phys. Rev. C{\bf 62}, 054904 (2000).

\bibitem{SPINO} P. Shukla and A. K. Mohanty, Phys. Rev. C{\bf 64},
        054910 (2001); O. Scavenius, A. Dumitru, 
        Phys. Rev. Lett. {\bf 83}, 4697 (1999).

\bibitem{ZABPRC} E.E. Zabrodin, L.V. Bravina, H. Stocker, and W. Griener,
         Phys. Rev. C{\bf 59}, 894 (1999).

\bibitem{THOMA} F. D. Steffen and M. H. Thoma, Phys. Lett. B510, 
                  1998 (2001).       
                  
\bibitem{GALL} K.\ Gallmeister, B.\ K\"ampfer, and O.P.\ Pavlenko, 
            Phys. rev. C{\bf 62}, 057901 (2000).

\end{thebibliography}
\end{document}




