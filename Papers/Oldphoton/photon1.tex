\documentstyle[aps,preprint,psfig]{revtex}

\renewcommand{\arraystretch}{1.25}
\newcommand{\figcaptionwidth}{15.cm}
\newcommand{\tabcaptionwidth}{13.75cm}

\newcommand{\be}{\begin{equation}}
\newcommand{\ee}{\end{equation}}

\newcommand{\bea}{\begin{eqnarray}}
\newcommand{\eea}{\end{eqnarray}}

\newcommand{\benn}{\begin{displaymath}}
\newcommand{\eenn}{\end{displaymath}}

\newcommand{\beann}{\begin{eqnarray*}}
\newcommand{\eeann}{\end{eqnarray*}}

\newcommand{\befig}{\begin{figure}}
\newcommand{\efig}{\end{figure}}

\newcommand{\inv}{\frac{1}}

\newcommand{\fm}{\mbox{fm}}
\newcommand{\MeV}{\mbox{MeV}}
\newcommand{\GeV}{\mbox{GeV}}
\newcommand{\TeV}{\mbox{TeV}}

\newcommand{\dhline}{\hline\hline}



\begin{document}

\begin{center}
{\large \bf Dynamics of quark-hadron phase transition
            and SPS photon and dilepton data}\\
{Prashant Shukla}\\
{\it Nuclear Physics Division,\\
 Bhabha Atomic Research Center, Mumbai 400085, India}
\end{center}

\begin{abstract}

   The photon and dilepton distributions have been calculated 
asuuming quark gluon plasma (QGP) formed at CERN energy which undergoes 
a first order phase transition to hadron phase. Both photon and dilepton 
data at CERN are found to be consistent with scenario of a fast
hadronization which coincides with freeze out.
  
  
\noindent
{\it Keywords}: Quark-Gluon Plasma,
Relativistic Heavy-Ion Collisions, Photon Production, 
Dilepton Production

\medskip

\noindent
{\it PACS numbers}: 12.38.Mh, 25.75.-q, 05.70.Fh

\end{abstract}

\section{Introduction} \label{Introduction}
   The motivation behind colliding heavy ions at relativistic
energies is to study the properties of strongly interacting matter at
high temperature where a phase transition to Quark Gluon Plasma is 
expected. With this aim, Pb+Pb collision experiments at
$\sqrt(s)=17.6A$ GEV have been carried out at CERN SPS giving a 
variety of data on electromagnetic and hadronic observables
[For a review see \cite{ESKOLA}]. 
   The most interesting are the electromagtic
probes \cite{WA98_2000,CERES}, which are sensitive to the various stages 
of the evolution of the system and may contain the information about
QGP formation. A considerbale theoretical work has been done 
to dig out the information on QGP from these 
data [For review see~\cite{GALER,PEITZ}].
   One generally assumes, the formation of quark gluon plasma (QGP),
which expands and cools down until it reaches a critical temperature $T_C$ 
where a phase transition from the quark matter to the hadron matter begins.
  In the most idealized picture, the temperature of the system is held
fixed at $T=T_C$ until the phase conversion is completely over. 
Such a picture is has been widely used in literature when describing
the electromagntic signals \cite{}.
  This is a good approximation when the expansion time scales are far 
greater than the phase transition time scales as in the case of early 
universe. For the case of relativistic heavy ion collisions, there have
been extensive studies of the space time evolution of 
QGP undergoing a first order phase transition
\cite{CSER,ZABPRC,SHUK,INHOMO}.
  As a result of rapid expansion of the system, a substantial amount of 
supercooling is expected and the dynamics of quark hadron phase 
transition becomes quite different in RHIC as compared to that 
in early universe \cite{INHOMO,AKM}. 
  It is expected that the QGP will supercool upto some temperature $T_m$ 
until the density of nucleated hadron
bubbles become sufficient to heat up the medium due to the release of latent
heat. Another possibility in case of strong enough supercooling is, 
the barrier between the metastable QGP and the stable hadron phase minima 
completely vanishes leading to a point of inflection at $T=T_S$ known as 
spinodal instability \cite{SPINO}. In this case, the rapidly quenched system 
leaves the region of metastability and enters the highly unstable spinodal 
region before a substantial amount of nucleation begins.
   The spinodal decomposition has also been suggested as a
possible mechanism of phase conversion for a rapidly expanding system of 
quarks and gluons \cite{SPINO,DUMHEP,DUMPRL}. This hadronization process
is much faster than the nucleation and is  
also  referred as explosive decomposition \cite{DUMEXP} as the particles 
formed would leave the system just after hadronization. 
 Such a picture is not new and has been presented in literature
as 'a fast shock like hadronization' in 
Refs.~\cite{CSORGO} and a 'sudden hadronization scenario' in 
Ref.~\cite{RAFEL}.
    The short life time observed at SPS and RHIC
may be indicative that the phase transition has gone through spinodal 
decomposition which coincides with freeze out.
 Recently, a fast shock like hadronization scenario is assumed 
to explain the HBT data \cite{CSERNAI}.

  The Photons and dileptons are the best probes of space time evolution of 
quark-gluon plasma (QGP) possibly produced in relativistic heavy-ion
collisions. Direct photon spectra have been measured by
the WA98 collaboration at the SPS~\cite{WA98_2000}.
The dilepton invariant mass spectrum has been measured by CERES \cite{CERES}
also at SPS.
   In the present work, the contribution of thermal photons and dileptons 
are calculated and compared with SPS data, assuming the QGP formation which 
expands and goes through a sudden hadronization coinciding with freeze out.
 Various other sources of photons and dileptons have also been added
in this work.
 In photon sector, the contribution of prompt photons due to the 
hard scattering of initial partons has been added which are calculated by QCD
while photons from the decay of hadrons
after freeze-out are already subtracted in the experimental analysis.
 Similarly the dilepton spectrum is the total of thermal dileptons 
integrated over space time and the background due to the
dileptons coming from initial scattering and the hadron decays after 
freeze out.


\section{Model}

 We consider that QGP is formed at some initial temperature $T_i$ and
initial time $\tau_i$ and cools upto freeze out temperature 
$T_f$ till time $\tau_f$ by the Bjorken
1+1-hydrodynamical expansion. Considering the short life time 
$\tau_f$ of the system, this is a good approximation and we need 
not consider any transeverse expansion.

  The total thermal photon/dilepton emission rate is given general by 
the sum of corresponding rates $R_{\rm QGP}$ in the QGP and $R_{\rm hadron}$ in the hadron regions 
integrated over space time volume element $d^4x = \tau d\tau d\eta \pi R^2$ as
\begin{eqnarray}\label{}
R_I  &=& \int_{\tau_i}^{\tau_f} \int_{-l}^{l} 
      \left[(1-h(\tau)) R_{\rm QGP}(T(\tau),\eta) 
   + h(\tau)  R_{\rm hadron}(T(\tau),\eta) \right]\tau d\tau d\eta \pi R^2
\end{eqnarray}
Here $\eta$ is the fluid rapidity and $R$ is the size of colliding 
nuclei given by $R=1.2\, A^{1/3}$. $h(\tau)$ is the hadronic fraction
at a time $\tau$ which we take as 0. 

 The temperature $T(\tau)$ and the hadronic fractions are
obtained from the hydrodynamic equations. The temperature variation
in the quark phase is governed by,
\begin{eqnarray}\label{}
T(\tau) = T_i \left({\tau_i \over \tau}\right)^{1/3}
\end{eqnarray}

 Assuming, Bjorken scaling, the experimentally observed rapidity 
density $dN/dy$ can be related to the initial conditions by

\begin{equation}
dN/dy = \pi R_A^2 4 a_q T_i^3 \tau_i/3.6,
\end{equation}
where $a_q$ is the number of degrees of freedom in quark phase.
The experimentally observed $dN/dy=1.5\, dN_{ch}/dy$ 


\section{The photon distribution}

  The production rate for hard thermal photons from an equilibrated 
QGP has been calculated in
perturbative thermal QCD applying the hard thermal loop (HTL)
resummation to account for medium effects. 


The {\em Compton scattering} and {\em $q\bar{q}$-annihilation} contribution
derived from the 1-loop HTL photon-polarization 
tensor~\cite{KAPUSTA_1991,BAIER_1991,TRAXLER_1995} is given by
%
\be
 E\,\frac{dN}{d^4x\,d^3p} \, =  {5\over 18 \pi^2}\,\alpha \alpha_s 
       \ln \left(\frac{0.23\,E}{\alpha_s\,T}\right)
            \,T^2\,e^{-E/T},
\label{1-loop}
\ee
%
The contributions from {\em bremsstrahlung}, 
and {\em $q\bar{q}$-annihilation with an additional scattering in the medium},
obtained from the 2-loop HTL photon-polarization
tensor~\cite{AURENCHE_1998,THOMA} are given by
%
\be
   E\,\frac{dN}{d^4x\,d^3p} \, = 
        0.0219\,\alpha \alpha_s
        \,T^2\,e^{-E/T},
\label{bremss}
\ee
%

%
\be
      E\,\frac{dN}{d^4x\,d^3p} \, =  
        0.0105\,\alpha \alpha_s
        \,E\,T\,e^{-E/T},
\label{qqbar-aws}
\ee
%

All three rates are listed for a two-flavored
($N_f = 2$) QGP. Here $\alpha=1/137$ and the strong coupling constant is 
$\alpha_s= 6\pi/((33-2N_f) \ln(8T/T_C))$ \cite{KARSCH}.
The rates~(\ref{bremss}) and~(\ref{qqbar-aws}) were 
{\em erroneously} multiplied 
by a factor of 4 \cite{AURENCHE_1998}, 
we take the rates in correct form \cite{THOMA}.

 The thermal photon production in an equilibrated hadron phase
is determined by considering various meson interactions
$\pi \pi \rightarrow \rho \gamma$, $\pi \rho \rightarrow \pi
\gamma$, $\omega \rightarrow \pi \gamma$ and $\rho
\rightarrow \pi \pi \gamma$~\cite{KAPUSTA_1991,NADEAU_1992}.  By considering
additionally the $\pi \rho \rightarrow a_1 \rightarrow \pi \gamma$ reaction, a
strong enhancement of the rate was observed~\cite{XIONG_1992}. 
 The numerical results of a detailed study of these dacays has been
fitted by analytical expression as a function of temperature and photon
energy as reported in Ref.~\cite{THOMA} given as
%
\be
        E\,\frac{dN}{d^4x\,d^3p} \, = 
        4.8\,T^{2.15}\,e^{-1/(1.35\,T\,E)^{0.77}}\,e^{-E/T},
\label{Markus_suggestion}
\ee
%
where photon energy~$E$ and temperature~$T$ are to be given in GeV to obtain the
rate in units of $\fm^{-4}\GeV^{-2}$. 

{\bf Prompt Photons}. 
 Dumitru et al. \cite{DUMI} estimate the contribution of prompt photons 
which employs the pQCD with inclusion of the effect of intrinsic 
transverse momentum of the partons. Their results for Pb+Pb collision
$\sqrt{s} = 17.4 A$GeV for $<K_T^2>=1.8$ GeV$^2$ including additional 
broadening from nuclear effects can be parameterized as 
\begin{equation}\label{promp}
 E \frac{d^3 N}{d^3 p} = \exp(a + bP_T + cP_T^2 + dP_T^3)
\end{equation}
where $a=-4.1506$, $b=-1.9845$, $c=0.0744$, and 
$d=-0.0383$.

 In the work of Gallmeister~\cite{GALLMEISTER_2000} hard photon 
 yield is generated by the event generator PYTHIA and are
parameterized same as Eq.~(\ref{promp}) with the coefficients
$a=-5.9883$, $b=2.0934$, $c=-1.6015$ and $d=0.1432$.
 The WA98 direct photon data analysis of Gallmeister et al.\ obtained in 
a model describing a {\em spherically} symmetric 
expansion~\cite{GALLMEISTER_2000}.

 It is also possible to reproduce the WA98 direct photon data analysis of
Srivastava et al.~\cite{SRIVASTAVA_2000}, which does not necessitate prompt
photons but instead initial conditions that are rather extreme for SPS, i.e.\ a
very small thermalization time of $\tau_0 = 0.2\;\fm$ and a very high initial
temperature of $T_0 = 335\;\MeV$. 
The effective degrees of freedom in the hadron gas has been increased
from the actual value of $g_h = 3$ to an effective one of $g_h =8$ in order 
to achieve the fit in the HHG thermal photon spectrum. This reduces
the life of mixed phase and hence reducing the contribution from 
hadron phase.


For {\em typical} parameters 
%
$\tau_i = 0.8\;\fm$, $T_c = 160\;\MeV$, $T_f = 140\;\MeV$, 
%
nucleon number $A=208$ (corresponding to Pb + Pb collisions)
we have obtained the photon spectrum.


\section{The dilepton distribution}

 The dilepton emission rate in the quark sector considering the 
processes ($q \bar q \rightarrow e^- e^+$)
is given \cite{KAJA,VOGT} by 

\begin{eqnarray}\label{}
 {dN \over d^4x d^2p_T dy dM^2}
&=& \frac{3}{(2\pi)^5} M^2 \sigma(M^2)
        \exp\left(-\frac{E}{T}\right), \nonumber \\
&=& \frac{\alpha^2}{8\pi^4} \left(\sum e_q^2\right) 
        \exp\left(-\frac{E}{T}\right).
\end{eqnarray}
Here, $\sigma(M^2)=4\pi\alpha^2/3M^2$ and $F_q = \sum e_q^2 = 5/9$.

The dilepton emission rate from hadron phase is given by
\begin{eqnarray}\label{}
\frac{dN}{d^4x d^2p_T dy dM^2} = \frac{\alpha^2}{8\pi^4}\, F_h \,
 \exp(-{E\over T}),
\end{eqnarray}
For the hadrons, if we assume 
$\pi\pi \rightarrow \rho \rightarrow l^+l^-$ is the dominant channel, 
\begin{eqnarray}\label{}
F_h = \frac{1}{12} \frac{m_\rho^4}{(m_\rho^2-M^2)^2 + m_\rho^2 \Gamma_\rho^2}
\end{eqnarray}
The parameters used are $m_\rho$=768 MeV, $\Gamma$=151 MeV. The detail
form factors can be obtained from Ref.~\cite{GALE}

We get, the $p_T$ distribution as
\begin{eqnarray}\label{}
\frac{dN}{d^4x dy dM dp_T d\phi} = \frac{\alpha^2}{4\pi^4}  F \, M \,
   \exp\left(-\frac{ \sqrt{M^2 + p_T^2} cosh (y-\eta)}{T}\right) \, p_T
\end{eqnarray}

and the invariant mass distribution 
\begin{eqnarray}\label{}
\frac{dN}{d^4x dy dM} = \frac{\alpha^2}{2\pi^3} F \, M^3 \,
\left({1\over x^2} + {1\over x}\right) \exp(-x),
\end{eqnarray}
where

\begin{eqnarray}\label{}
x=\frac{ M \cosh (y-\eta)}{T}.
\end{eqnarray}

Here $M$, $p_T$ and $y$ are the mass, transverse momentum,
and  rapidity of the lepton pair and $\eta$ is the rapidity
of the fluid with temperature $T$.



\section{HBT radii and other parameters at SPS and RHIC}

At SPS 158A GeV Pb + Pb collisions at center of mass energy 
17.5A GeV the total charged particle multiplicity is 
$N_{ch}= 700$. The flow parameters are $\eta \simeq \pm 1$ and
$\beta_c$ = 0.76 \cite{LET}. The freeze out parameters are
$\mu_B$= 270 MeV  and $T_f$ = 143 MeV \\

At RHIC Au + Au collisions at center of mass energy 
130A GeV the total charged particle multiplicity is 
$N_{ch}= 4000$. The flow parameters are $\eta \simeq \pm 1.5$,
$\beta_c$ = 0.90. The freeze out parameters are
$\mu_B$= 50 MeV, $T_f$ = 140 MeV.
The HBT radii from PHENIX experiment \cite{PHEN} are 5-7 fm.
The ratio $R_{out}/R{side} \leq 1$. 
On the basis of hydrodynamics $R_{out}/R{side} \sim 2$.
\ \\

\newpage

\begin{figure}
\centerline{\psfig{figure=photonew.ps,width=10cm,height=9.5cm}}
\caption{}
\label{photon}
\end{figure}

\begin{figure}
\centerline{\psfig{figure=ceres.ps,width=10cm,height=9.5cm}}
\caption{}
\label{photon}
\end{figure}


\begin{thebibliography}{00}

\bibitem{ESKOLA} {\it High Energy Nuclear Collisions}, Plenary talk 
  given at Int. Europhysics conference on High Energy Physics 
  (EPS-HEP99), Tampere, Finland, July 15-21, 1999, 
  Preprint: hep-ph/9911350 (1999).
  
\bibitem{GALER} C. Gale, in {Quark Gluon Plasma 3}, World Scientific,
         Singapore, 2003. 
         
\bibitem{PEITZ} For recent review,  T. Peitzman, nucl-ex/0201003 (2002).

\bibitem{}

\bibitem{WA98_2000} WA98 Collab., M.M.\ Aggarwal et al., 
            Phys. Rev. Lett. 85, 3598 (2000); nucl-ex/0006007 (2000).

\bibitem{CERES} B. Lenkeit for CERES Coll., 
        Nucl. Phys. {\bf A654}, 627c (1999); nucl-ex/9910015.

\bibitem{CSER} L.P. Csernai and J.I. Kapusta, Phys. Rev. Lett. {\bf 69},
               737 (1992).

\bibitem{ZABPRC} E.E. Zabrodin, L.V. Bravina, H. Stocker, and W. Griener,
         Phys. Rev. C{\bf 59}, 894 (1999).

\bibitem{SHUK} P. Shukla, S.K. Gupta, and A.K. Mohanty,
         Phys. Rev. C{\bf 59}, 914 (1999); {\it ibid} {\bf 62}, 39901 (2000).

\bibitem{INHOMO} P. Shukla, A.K. Mohanty, S. K. Gupta, and M. Gleiser,
              Phys. Rev. C{\bf 62}, 054904 (2000).

\bibitem{AKM} A.K. Mohanty, P. Shukla and M. Gleiser,
              Phys. Rev. C{\bf 65}, 034908 (2002).
              
\bibitem{SPINO} P. Shukla and A. K. Mohanty, Phys. Rev. C{\bf 64},
        054910 (2001).              

\bibitem{DUMHEP} O. Scavenius, A. Dumitru, E.S. Fraga, J.T. Lenaghan,
         A.D. Jackson, Phys. Rev. D{\bf 63}, 116003 (2001).

\bibitem{DUMPRL} O. Scavenius, A. Dumitru,
         Phys. Rev. Lett. {\bf 83}, 4697 (1999).
         
\bibitem{DUMEXP} O. Scavenius, A. Dumitru, A.D. Jackson, 
          hep-ph/0103219.
          
\bibitem{CSORGO} T. Csorgo, L.P. Csernai, Phys. Lett {\bf B333}, 494 (1994);
       L.P. Csernai, I.N. Mishustin, Phys. Rev. Lett. {\bf 74}, 5005 (1995).
       
\bibitem{RAFEL} J. Rafelski and J. Letessier, Phys. Rev. Lett. {\bf 85},
                  4695 (2000).
         
\biitem{CSERNAI} L.P. Csernai, M.I. Gorenstein, L.L. jenkovszky, 
                        I. Lovas and V.K. Magas, hep-ph/0210297.


\bibitem{MAD} M.B. Christiansen and J. Madsen,
           Phys. Rev. D{\bf 53}, 5446 (1996).

\bibitem{KAPUSTA_1991} J.\ Kapusta, P.\ Lichard, and D.\ Seibert, 
   Phys.\ Rev.\ {\bf D44}, 2774 (1991); {\bf D47}, 4171 (1993). 

\bibitem{BAIER_1991} R.\ Baier, H.\ Nakkagawa, A.\ Ni\'{e}gawa, and
K.\ Redlich, Z.\ Phys.\ {\bf C53}, 433 (1992).

\bibitem{TRAXLER_1995} C.T.\ Traxler, H.\ Vija, and M.H.\ Thoma,
Phys.\ Lett.\ {\bf B346}, 329 (1995).
  
\bibitem{AURENCHE_1998} P.\ Aurenche, F.\ Gelis, R.\ Kobes, and H.\ Zaraket,
  Phys.\ Rev.\ {\bf D58}, 085003 (1998); \mbox{hep-ph/9804224}. 

\bibitem{THOMA} F. D. Steffen and M. H. Thoma, Phys. Lett. B510, 
                  1998 (2001).
                  
\bibitem{KARSCH} F. Karsch, Z. Phys. C{\bf 38}, 147 (1988).
  
\bibitem{NADEAU_1992} H.\ Nadeau, J.\ Kapusta, and P.\ Lichard,
Phys.\ Rev.\ {\bf C45}, 3034 (1992); {\bf C47} 2426 (1993).

\bibitem{XIONG_1992} L.\ Xiong, E.\ Shuryak, and G.E.\ Brown, Phys.\
Rev.\ {\bf D46}, 3798 (1992).

\bibitem{DUMI} A. Dumitru, L. Frankfurt, L. Geland, H. Stocker, and
               M. Strikeman, Phys. Rev. C{\bf 64}, 054909 (2001).

\bibitem{GALLMEISTER_2000} K.\ Gallmeister, B.\ K\"ampfer, and O.P.\ Pavlenko, 
            Phys. rev. C{\bf 62}, 057901 (2000); hep-ph/0006134 (2000).

\bibitem{SRIVASTAVA_SINHA_1999} D.K.\ Srivastava and B.C.\ Sinha,
             Eur.\ Phys.\ J.\ {\bf C12}, 109 (2000).

\bibitem{SRIVASTAVA_2000} D.K.\ Srivastava and B.\ Sinha, nucl-th/0006018 (2000).

\bibitem{KAJA} K. Kajantie, M. Kataja, L. McLerran, and P. V. Ruuskanen,
               Phys. Rev. D{\bf 34}, 811 (1986).

\bibitem{VOGT} R. Vogt, B. V. Jacak, P. L. McGaughey, P. V. Ruuskanen,
          Phys. Rev. D {\bf 49}, 3345 (1994).
          
\bibitem{GALE} C. Gale and P. Lichard, Phys. Rev. D{\bf 49}, 3338 (1994);
               C. Song, C.M. Ko, and C. Gale, {\it ibit} 50 R 1827 (1994).

\bibitem{LET} J. Letessier and J. Rafelski, hep-ph/0106151.

\bibitem{PHEN} PHENIX Collaboration, K. Adcox et. al., 
                    nucl-ex/0109003.

\end{thebibliography}

\end{document}




