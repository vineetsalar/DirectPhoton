\documentstyle[12pt,psfig]{article}
\textwidth=14.5cm
\textheight=20cm
\pagestyle{empty}
\begin{document}

\begin{center}
{\large \bf A first order quark-hadron phase transition and 
 CERN SPS photon spectrum}\\
Prashant Shukla  \\
{\it Nuclear Physics Division, 
Bhabha Atomic Research Centre,\\
Trombay, Mumbai 400 085, India}\\
\end{center}

In this work, the photon spectrum [1] measured in Pb+Pb collision 
experiments at $\sqrt s = 17.6A$ GeV at SPS, CERN has been analyzed using 
different scenarios of a first order quark-hadron phase transition
in a hydrodynamical model.
  In hydrodynamical models, one assumes that QGP is formed with given 
initial conditions which then expands till it reaches the critical 
temperature $T_C$ for quark-hadron phase transition.
  In the idealized Maxwell construction, the temperature of the system 
is held fixed at $T=T_C$ until the hadronization is completed. 
Thus, in turn one assume that the latent heat obtained by the phase 
transition compensates the temperature drop due to cooling.
Such a picture has been widely used in literature when describing
the electromagnetic signals.
 For the case of relativistic heavy ion collisions, when the
hydrodynamical equations are coupled with nucleation rate equation, 
the QGP is shown to supercool to a temperature $T_S$ lower than 
$T_C$ [2] as shown in Fig.~1. The hadronization proceeds by 
thermal fluctuation from this metastable QGP phase to stable hadron phase. 
The system reheats due to the release of latent heat and approaches to 
$T_C$ as the hadronization proceeds.

It is also possible that QGP supercools to a temperature where it goes 
to an unstable state then one can not define the phase transition rate by 
thermal fluctuation. This point is known as spinodal instability [3].
 The QGP will then break up into small droplets of plasma,
which will form hadrons. 
 This occurs in a short time and the system may not be able to undergo 
further reequilibration and would freeze out at this point. 

  When most of the QGP volume has been converted to hadrons, 
the already nucleated hadron bubbles would tend to join together and 
the QGP droplets will be trapped inside. At such a point, fluctuation 
theory is no more applicable. The system is highly unstable at this point 
and may simply breaks up into fragments: hadronic clusters and small droplets 
of plasma which then form the final state hadrons [4]. 

\begin{figure}[!htb]
\centerline{\hbox{\psfig{figure=supern.ps,height=6cm,width=7cm} 
\psfig{figure=photodif.ps,height=6cm,width=7cm}}}
\end{figure} 

  As all these scenarios exist in the literature, it is worthwhile 
to see their manifestation in the measured photon spectrum,
assuming that QGP cools by Bjorken hydrodynamics. 
To get the equations of state it is assumed that QGP is a massless 
2 flavour parton gas and hadronic phase is a massless three 
pion gas.
We define following four scenario for the calculations:


(1) Idealized Maxwell construction (from $T_i$ to $T_f$.)
(2) Supercooling scenario (from $T_i$ to $T_f$.)
(3) Spinodal decomposition followed by freeze out
             (from $T_i$ to $T_S$.)
(4) Fragmentation at 95 \% of hadronization followed by freeze out
               (from $T_i$ to $T_P$.)

  The total thermal photon emission rate is given by the sum of 
corresponding rates in the QGP and in the hadron regions integrated 
over space time volume obtained from above four scenario. The photon 
production rates are taken from the parameterized forms of Ref.~[5].
The contribution of prompt photons has been obtained by the PYTHIA
calculations.

   Figure~2 demonstrates how the thermal photons contributions due to 
different evolutions compare with SPS data. 
From this figure, scenario (3) seems the most appropriate evolution scenario. 
 It must be noted that these calculations depend on the initial conditions, 
equation of state, photon rates and hydrodynamical expansion but, would 
certainly be helpful to understand the various evolution scenarios and their 
manifestation in photon spectrum.\\

\noindent
{\bf 1.} WA98 Collab., M.M.\ Aggarwal et al., 
            Phys. Rev. Lett. 85, 3598 (2000).\\
{\bf 2.} P. Shukla, A.K. Mohanty, S. K. Gupta, and M. Gleiser,
              Phys. Rev. C{\bf 62}, 054904 (2000).\\
{\bf 3.} P. Shukla and A. K. Mohanty, Phys. Rev. C{\bf 64},
        054910 (2001); O. Scavenius, A. Dumitru, 
        Phys. Rev. Lett. {\bf 83}, 4697 (1999).\\
{\bf 4.} E.E. Zabrodin, L.V. Bravina, H. Stocker, and W. Griener,
         Phys. Rev. C{\bf 59}, 894 (1999).\\
{\bf 5.} F. D. Steffen and M. H. Thoma, Phys. Lett. B510, 
               1998 (2001).\\
\end{document}
