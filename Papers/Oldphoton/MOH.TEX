Date: Tue, 30 Sep 2003 10:35:46 +0530 (IST)
From: Prashant Shukla <pshukla@apsara.barc.ernet.in>
To: Ajit Kumar Mohanty <Ajit.Kumar.Mohanty@cern.ch>
Subject: Re: your mail

Dear Ajit,
  Thank you for your comments. I think these will help in improving the
quality of paper. I will see how to include them in a best possible way.

> It is just for my
> clarification. If I undestand correctly, the 2 and 4 scenario are nearly
> equivalennt except for the fact that \tau_p is a variable and you need to
> allow hadronisation to start while the system itself is in the mixed phase.

   Same supercooling curve is used for both the scenario and freeze out
is implimented at different times. First scenario corresponds to
hadronization and freeze out due to spinodal instability. The second
corresponds to fragmentation when 95% hadronization is over.
I will make it clear in the paper.

>Then can you say at some point that nucleation feature is essential
> without which the data can not be explained with the set of parameters
> which you have used (i.e. reasonable T and t=0.8 fm).

  The nucleation feature is essential anyway as the first order phase
transition can not take place at TC. I will stress this point.
The main thing in fitting photon spectrum is to reduce the contribution
in low PT and increase in high PT. So even without nucleation one can
have a good fit to data by simultaneously doing these.
 1. Use transverse expansion and use rich eqn of state both reduce the
 lifetime of mixed phase and contribution in low PT
 2. Increase the temp. little more. It increases the slope.
 With this they can not have a say about the shorter freeze out time
obtained by HBT.

> For example, is it
> possible to get the similar behaviour even under ideal costruction as used
> by others if one assumes the medium is viscous.
  No. You will enhance photons in low PT region by including viscosity.

> All that you need is to
> allow hadronisation when 80 to 90 percent of matter has been converted
> into hadron phase (?). I know this sounds ridiculos !!. Probably, in
> order the hadronisation to be possible one need to remain below T_c,
> (in a metastable phase) so that freeze out can begin when h reaches
> 90 percent.

  Yes. You should have some physical ground on which you freeze out.
For example I can say hadron bubbles and few QGP bubbles which themselves
are like hadrons and then freeze out. They can use this terminology.

  I realize the best would be to do  (3+1) expansion. These cal
will be still valid but only the numbers will change. Presently,
I will say that these the comparative contributions due to different
scenarios and not much stress on the fitting.

 Regards,
Prashant



